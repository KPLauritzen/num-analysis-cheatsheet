\documentclass[10pt,landscape]{article}
\usepackage{multicol}
\usepackage{calc}
\usepackage{ifthen}
\usepackage[landscape]{geometry}
\usepackage{amsmath,amsthm,amsfonts,amssymb}
\usepackage{color,graphicx,overpic}
\usepackage{hyperref}

% This sets page margins to .5 inch if using letter paper, and to 1cm
% if using A4 paper. (This probably isn't strictly necessary.)
% If using another size paper, use default 1cm margins.
\ifthenelse{\lengthtest { \paperwidth = 11in}}
    { \geometry{top=.5in,left=.5in,right=.5in,bottom=.5in} }
    {\ifthenelse{ \lengthtest{ \paperwidth = 297mm}}
        {\geometry{top=1cm,left=1cm,right=1cm,bottom=1cm} }
        {\geometry{top=1cm,left=1cm,right=1cm,bottom=1cm} }
    }

% Turn off header and footer
\pagestyle{empty}

% Redefine section commands to use less space
\makeatletter
\renewcommand{\section}{\@startsection{section}{1}{0mm}%
                                {-1ex plus -.5ex minus -.2ex}%
                                {0.5ex plus .2ex}%x
                                {\normalfont\large\bfseries}}
\renewcommand{\subsection}{\@startsection{subsection}{2}{0mm}%
                                {-1explus -.5ex minus -.2ex}%
                                {0.5ex plus .2ex}%
                                {\normalfont\normalsize\bfseries}}
\renewcommand{\subsubsection}{\@startsection{subsubsection}{3}{0mm}%
                                {-1ex plus -.5ex minus -.2ex}%
                                {1ex plus .2ex}%
                                {\normalfont\small\bfseries}}
\makeatother

% Define BibTeX command
\def\BibTeX{{\rm B\kern-.05em{\sc i\kern-.025em b}\kern-.08em
    T\kern-.1667em\lower.7ex\hbox{E}\kern-.125emX}}

% Don't print section numbers
\setcounter{secnumdepth}{0}


\setlength{\parindent}{0pt}
\setlength{\parskip}{0pt plus 0.5ex}

%My Environments
\newtheorem{example}[section]{Example}
% -----------------------------------------------------------------------

\begin{document}
\raggedright
\footnotesize
\begin{multicols}{3}


% multicol parameters
% These lengths are set only within the two main columns
%\setlength{\columnseprule}{0.25pt}
\setlength{\premulticols}{1pt}
\setlength{\postmulticols}{1pt}
\setlength{\multicolsep}{1pt}
\setlength{\columnsep}{2pt}

\begin{center}
     \Large{\underline{MATH104A - gl;hf}} \\
\end{center}

\section{Cubic Splines}
\begin{enumerate}
\item Piecewise cubic polynomial: 
$S_{j} = a_{j} + b_{j}(x-x_{j}) + c_{j}(x-x_{j})^{2} +
d_{j}(x-x_{j})^{3}$.
\item Continuous: $C[x_{0},x_{n}], S_{j+1}(x_{j+1}) = S_{j}(x_{j+1})$. 
\item 1st derivative is cont.: $C^{1}[x_{0},x_{n}], S_{j+1}'(x_{j+1}) =
S_{j}'(x_{j+1})$.
\item 2nd derivative is cont.: $C^{2}[x_{0},x_{n}], S_{j+1}''(x_{j+1}) =
S_{j}''(x_{j+1})$.
\item Interpolating: $S_{j}(x_{j}) = f(x_{j})$
\item Boundary conditions: Natural $S''(x_{0}) = S''(x_{n}) = 0$,
or Clamped $S'(x_{0}) = f'(x_{0})$ and $S'(x_{n}) = f'(x_{n})$
\end{enumerate}

\section{Numerical Differentiation}
Generally, a $(n+1)$-point formula to approximate $f'(x_{j})$:
\begin{align*}
  f'(x_{j}) = \sum_{k=0}^{n}f(x_{k}) L_{k}'(x_{j}) + 
  \frac{f^{(n+1)}(\xi(x_{j}))}{(n+1)!} \prod_{k=0, k\neq j}^{n}(x_{j}
  - x_{k})
\end{align*}

\textbf{For n=1}: Forward difference
\begin{align*}
  f'(x) \approx \frac{f(x+h) - f(x)}{h}
\end{align*}
Backwards difference
\begin{align*}
  f'(x) \approx  \frac{f(x) -f(x-h)}{h}
\end{align*}
Central difference
\begin{align*}
  f'(x) \approx \frac{f(x+h) - f(x-h)}{2h}
\end{align*}

\textbf{For n=2}
Central
\begin{align*}
  f''(x) \approx \frac{f(x+h) - 2f(x) + f(x-h)}{h^{2}}
\end{align*}

\section{Numerical Integration}
\subsection{Simple Quadrature}
\textbf{Trapezoidal Rule}
\begin{align*}
  \int_{a}^{b}f(x)dx = \frac{h}{2}[f(a) + f(b)] -
  \underbrace{\frac{h^{3}}{12}f''(\xi)}_{\text{error}}
\end{align*}
with $h = b-a$

\textbf{Simpson's Rule}
\begin{align*}
  \int_{a}^{b}f(x)dx = \frac{h}{3}[f(a) +4f(x_{1}) + f(b)] - \underbrace{\frac{h^{5}}{90}f^{(4)}(\xi)}_{\text{error}}  
\end{align*}
with $h=(b-a)/2$, $x_{1}=a+h$.
\subsection{Degree of Accuracy, Precision}
A quadrature is of precision $m$ if it is exact for all polynomials of
degree $k \leq m$. Or Precision(quad) $=m$ iff quad is exact for
$x^{k} \quad k=0,1,\ldots > m$.

\subsection{Composite Integration}
\textbf{Trapezoidal}:
\begin{align*}
  \int_{a}^{b}f(x)dx = \frac{h}{2}[f(a) + f(b) + 2 \sum_{j=1}^{n}f(x_{j})] -
  \underbrace{\frac{h^{3}}{12}f''(\xi)}_{\text{error}}
\end{align*}
with $h=(b-a)/n$.

\textbf{Simpsons}:
\begin{align*}
\int_a^b f(x) \, dx\approx 
\frac{h}{3}\bigg[f(x_0)+2\sum_{j=1}^{n/2-1}f(x_{2j})+
4\sum_{j=1}^{n/2}f(x_{2j-1})+f(x_n)
\bigg]
\end{align*}
with $h=(b-a)/n$ and $x_{j} = a +jh$ for $j=0,1,\ldots,
n-1$. ($x_{0}=a, x_{n}=b)$. The error is
\begin{align*}
  \frac{h^4}{180}(b-a) \max_{\xi\in[a,b]} |f^{(4)}(\xi)|
\end{align*}

Can also be written as
\begin{align*}
\int_a^b f(x) \, dx\approx
\frac{h}{3}[f(x_{0})+4f(x_1)+2f(x_2)+4f(x_3)+\cdots+f(x_n)]
\end{align*}

\section{Truncation Error}
For $w_{0} = \alpha$ and $w_{i+1} = w_{i} + h\phi(t_{i},w_{i})$ with
$i=0,1,\ldots, N-1$ the truncation error is
\begin{align*}
  \tau_{i+1}(h) = \frac{y(t_{i+1}) - y(t_{i})}{h} - \phi(t_{i},y(t_{i}),h)
\end{align*}
\section{Initial Value Problem}
\begin{align*}
  y' = f(t,y) \quad a<t<b \quad -\infty < y < \infty \quad y(a) = \alpha
\end{align*}
Solve analytically by integrating on both sides and applying initial
conditions. 
\textbf{Lipschitz}: Guarantees unique solution. 
$f$ is Lipschitz if
\begin{align*}
  \left|\frac{df}{dy}(t,y)\right| \leq L
\end{align*}

\textbf{Euler's Method}: We want an approx. of $y$ at $t_{n}$. $y_{n} \approx y(t_{n})$.
\begin{align*}
  y(t_{0}) = y_{0}; \quad
  y_{n+1} = y_{n} + h f(t_{n},y_{n})
\end{align*}
with $t_{n} = t_{0} + nh $

\textbf{Runge-Kutta Method} (of several orders?)
\begin{align*}
y_{n+1} &= y_n + \tfrac{1}{6} \left(k_1 + 2k_2 + 2k_3 + k_4 \right)\\
t_{n+1} &= t_n + h
\end{align*} with
\begin{align*}
k_1 &= hf(t_n, y_n),
&k_2 = hf(t_n + \tfrac{1}{2}h , y_n +  \tfrac{1}{2} k_1),
\\
k_3 &= hf(t_n + \tfrac{1}{2}h , y_n +   \tfrac{1}{2} k_2), 
&k_4 = hf(t_n + h , y_n + k_3).
\end{align*}
\textbf{AB}: Adams-Bashforth are explicit. Reaches order $s$ for $s$
steps. 

\textbf{AM}: Adams-Moulton are implicit. It removes one restriction,
and allows it to reach order $s+1$ for a $s$ step method. 

One-step methods vs Multi-step methods

\section{Richardson Extrapolation}
\begin{align*}
  N(h) = M + \underbrace{k_{1}h + k_{2}h^{2} + \ldots}_{\text{error}}
\end{align*}
Eliminate leading order of $h$. 
\begin{align*}
  N(h/2) &= M + \frac{1}{2}k_{1}h + \frac{1}{4}k_{2}h^{2} + \ldots
\\
2 N(h/2) - N(h) &= M + \hat{k_{2}}h^{2} + \ldots = N_{2}(h)
\end{align*}
General Formula:
\begin{align*}
  N_{i+1}(h) = \frac{t^{k_i} N_i(\tfrac{h}{t}) - N_i(h)}{t^{k_i} - 1}
\end{align*}


\end{multicols}
\end{document}
%%% Local Variables: 
%%% mode: latex
%%% TeX-master: t
%%% End: (x-x_{j})
