\documentclass[10pt,landscape]{article}
\usepackage{multicol}
\usepackage{calc}
\usepackage{ifthen}
\usepackage[landscape]{geometry}
\usepackage{amsmath,amsthm,amsfonts,amssymb}
\usepackage{color,graphicx,overpic}
\usepackage{hyperref}

% This sets page margins to .5 inch if using letter paper, and to 1cm
% if using A4 paper. (This probably isn't strictly necessary.)
% If using another size paper, use default 1cm margins.
\ifthenelse{\lengthtest { \paperwidth = 11in}}
    { \geometry{top=.5in,left=.5in,right=.5in,bottom=.5in} }
    {\ifthenelse{ \lengthtest{ \paperwidth = 297mm}}
        {\geometry{top=1cm,left=1cm,right=1cm,bottom=1cm} }
        {\geometry{top=1cm,left=1cm,right=1cm,bottom=1cm} }
    }

% Turn off header and footer
\pagestyle{empty}

% Redefine section commands to use less space
\makeatletter
\renewcommand{\section}{\@startsection{section}{1}{0mm}%
                                {-1ex plus -.5ex minus -.2ex}%
                                {0.5ex plus .2ex}%x
                                {\normalfont\large\bfseries}}
\renewcommand{\subsection}{\@startsection{subsection}{2}{0mm}%
                                {-1explus -.5ex minus -.2ex}%
                                {0.5ex plus .2ex}%
                                {\normalfont\normalsize\bfseries}}
\renewcommand{\subsubsection}{\@startsection{subsubsection}{3}{0mm}%
                                {-1ex plus -.5ex minus -.2ex}%
                                {1ex plus .2ex}%
                                {\normalfont\small\bfseries}}
\makeatother

% Define BibTeX command
\def\BibTeX{{\rm B\kern-.05em{\sc i\kern-.025em b}\kern-.08em
    T\kern-.1667em\lower.7ex\hbox{E}\kern-.125emX}}

% Don't print section numbers
\setcounter{secnumdepth}{0}


\setlength{\parindent}{0pt}
\setlength{\parskip}{0pt plus 0.5ex}

%My Environments
\newtheorem{example}[section]{Example}
% -----------------------------------------------------------------------

\begin{document}
\raggedright
\footnotesize
\begin{multicols}{3}


% multicol parameters
% These lengths are set only within the two main columns
%\setlength{\columnseprule}{0.25pt}
\setlength{\premulticols}{1pt}
\setlength{\postmulticols}{1pt}
\setlength{\multicolsep}{1pt}
\setlength{\columnsep}{2pt}

\begin{center}
     \Large{\underline{MATH104A - gl;hf}} \\
\end{center}

\section{Interpolation}

\subsection{Divided Difference}
Also centered?

\subsection{Cubic Splines}
Conditions:

\begin{enumerate}
\item Piecewise cubic polynomial: 
$S_{j} = a_{j} + b_{j}(x-x_{j}) + c_{j}(x-x_{j})^{2} +
d_{j}(x-x_{j})^{3}$.
\item Continuous: $C[x_{0},x_{n}], S_{j+1}(x_{j+1}) = S_{j}(x_{j+1})$. 
\item 1st derivative is cont.: $C^{1}[x_{0},x_{n}], S_{j+1}'(x_{j+1}) =
S_{j}'(x_{j+1})$.
\item 2nd derivative is cont.: $C^{2}[x_{0},x_{n}], S_{j+1}''(x_{j+1}) =
S_{j}''(x_{j+1})$.
\item Interpolating: $S_{j}(x_{j}) = f(x_{j})$
\item Boundary conditions: Natural $S''(x_{0}) = S''(x_{n}) = 0$,
or Clamped $S'(x_{0}) = f'(x_{0})$ and $S'(x_{n}) = f'(x_{n})$
\end{enumerate}

\section{Numerical Differentiation}
Generally, a $(n+1)$-point formula to approximate $f'(x_{j})$:
\begin{align*}
  f'(x_{j}) = \sum_{k=0}^{n}f(x_{k}) L_{k}'(x_{j}) + 
  \frac{f^{(n+1)}(\xi(x_{j}))}{(n+1)!} \prod_{k=0, k\neq j}^{n}(x_{j}
  - x_{k})
\end{align*}

Forward difference, backwards difference. 

n=1,2,4

\section{Numerical Integration}
\subsection{Trapezoidal Rule}
\begin{align*}
  \int_{a}^{b}f(x)dx = \frac{h}{2}[f(a) + f(b)] -
  \underbrace{\frac{h^{3}}{12}f''(\xi)}_{\text{error}}
\end{align*}
with $h = b-a$
\subsection{Simpson's Rule}
\begin{align*}
  \int_{a}^{b}f(x)dx = \frac{h}{3}[f(a) +4f(x_{1}) + f(b)] - \underbrace{\frac{h^{5}}{90}f^{(4)}(\xi)}_{\text{error}}  
\end{align*}
with $h=(b-a)/2$, $x_{1}=a+h$.
\subsection{Degree of Accuracy, Precision}
A quadrature is of precision $m$ if it is exact for all polynomials of
degree $k \leq m$. Or Precision(quad) $=m$ iff quad is exact for
$x^{k} \quad k=0,1,\ldots > m$.
\section{Truncation Error}

\section{Initial Value Problem}
Euler's Method

Runge-Kutta Method (of several orders?)

Adams-Bashforth

Adams-Moulton


\section{Richardson Extrapolation}
\begin{align*}
  N(h) = M + \underbrace{k_{1}h + k_{2}h^{2} + \ldots}_{\text{error}}
\end{align*}
Eliminate leading order of $h$. 
\begin{align*}
  N(h/2) &= M + \frac{1}{2}k_{1}h + \frac{1}{4}k_{2}h^{2} + \ldots
\\
2 N(h/2) - N(h) &= M + \hat{k_{2}}h^{2} + \ldots = N_{2}(h)
\end{align*}
Boris knows general formula (from wiki?)


\end{multicols}
\end{document}
%%% Local Variables: 
%%% mode: latex
%%% TeX-master: t
%%% End: 
